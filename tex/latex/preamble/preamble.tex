\usepackage{amsmath}
\usepackage{amssymb}
\usepackage{amsthm}
\usepackage{mathrsfs}
\usepackage{enumitem}
\usepackage[french]{babel}
\usepackage[utf8]{inputenc}
\usepackage{listings}
\usepackage{graphicx}
\usepackage[T1]{fontenc}
\usepackage{empheq}
\usepackage{pgf}
\usepackage{color}
\usepackage{siunitx}
\usepackage{tabularx}
\usepackage{stmaryrd}
\usepackage{pgfplots}
\usepackage{exercices}
\usepackage{tikz,tkz-tab}
\usepackage[european]{circuitikz}
\usepackage{lesson}
\usepackage{esvect}
\usepackage{fourier-orns}
\usepackage{blindtext}
\usepackage{float}
\usepackage{esint}
\usepackage[xindy]{glossaries}
\usepgfplotslibrary{units}

%Signe danger
\usepackage{stackengine}
\usepackage{scalerel}
\newcommand\dangersign[1][2ex]{%
  \renewcommand\stacktype{L}%
  \scaleto{\stackon[1.3pt]{\color{red}$\triangle$}{\tiny !}}{#1}%
}

\definecolor{mygreen}{rgb}{0,0.6,0}
\definecolor{mygray}{rgb}{0.5,0.5,0.5}
\definecolor{mymauve}{rgb}{0.58,0,0.82}

\lstset{ 
  backgroundcolor=\color{white},   % choose the background color; you must add \usepackage{color} or \usepackage{xcolor}; should come as last argument
  basicstyle=\footnotesize\ttfamily,        % the size of the fonts that are used for the code
  belowskip=0pt,
  breakatwhitespace=false,         % sets if automatic breaks should only happen at whitespace
  breaklines=true,                 % sets automatic line breaking
  captionpos=b,                    % sets the caption-position to bottom
  commentstyle=\color{mygreen},    % comment style
  deletekeywords={...},            % if you want to delete keywords from the given language
  escapeinside={\%*}{*)},          % if you want to add LaTeX within your code
  extendedchars=true,              % lets you use non-ASCII characters; for 8-bits encodings only, does not work with UTF-8
  frame=single,	                   % adds a frame around the code
  keepspaces=true,                 % keeps spaces in text, useful for keeping indentation of code (possibly needs columns=flexible)
  keywordstyle=\color{black},       % keyword style
  language=bash,                 % the language of the code
  morekeywords={*,...},            % if you want to add more keywords to the set
  numbers=left,                    % where to put the line-numbers; possible values are (none, left, right)
  numbersep=5pt,                   % how far the line-numbers are from the code
  numberstyle=\tiny\color{mygray}, % the style that is used for the line-numbers
  rulecolor=\color{black},         % if not set, the frame-color may be changed on line-breaks within not-black text (e.g. comments (green here))
  showspaces=false,                % show spaces everywhere adding particular underscores; it overrides 'showstringspaces'
  showstringspaces=false,          % underline spaces within strings only
  showtabs=false,                  % show tabs within strings adding particular underscores
  stepnumber=0,                    % the step between two line-numbers. If it's 1, each line will be numbered
  stringstyle=\color{black},     % string literal style
  tabsize=2,	                   % sets default tabsize to 2 spaces
  title=\lstname,                   % show the filename of files included with \lstinputlisting; also try caption instead of title
  inputencoding=utf8,
  extendedchars=true
}

\tikzset{
  oplus/.style={path picture={%
	  \draw[black]
    (path picture bounding box.south west) -- (path picture bounding box.north east)
    (path picture bounding box.north west) -- (path picture bounding box.south east);
}}} 


\tikzset{ 
  xmin/.store in=\xmin, xmin/.default=-3, xmin=-3, 
  xmax/.store in=\xmax, xmax/.default=3, xmax=3, 
  ymin/.store in=\ymin, ymin/.default=-3, ymin=-3, 
  ymax/.store in=\ymax, ymax/.default=3, ymax=3, 
}

% Commande qui trace la grille entre (xmin,ymin) et (xmax,ymax) 
\newcommand {\grille} 
{\draw[help lines] (\xmin,\ymin) grid (\xmax,\ymax);} 
% Commande \axes 
\newcommand {\axes} { 
\draw[->] (\xmin,0) -- (\xmax,0); 
\draw[->] (0,\ymin) -- (0,\ymax); 
} 
% Commande qui limite l'affichage à (xmin,ymin) et (xmax,ymax) 
\newcommand {\fenetre} 
{\clip (\xmin,\ymin) rectangle (\xmax,\ymax);} 

\usepgfplotslibrary{fillbetween}

\pgfplotsset{
    standard/.style={
        axis x line=middle,
        axis y line=middle,
        every axis x label/.style={at={(current axis.right of origin)},anchor=north west},
        every axis y label/.style={at={(current axis.above origin)},anchor=north east}
    }
}
 
\usetikzlibrary{3d,arrows,arrows.meta,babel,decorations.markings,decorations.pathmorphing,decorations.shapes,patterns,backgrounds,positioning,fit,petri}
%\SIstyle{German}
\selectlanguage{french}
